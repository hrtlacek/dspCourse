\chapter{Modulation}
\label{Modulation}

\section{Notizen}

kürzer. Modulation als sub einheit einplanen.
Sounddesign challg. halbe stunde ok..

Passt garnicht: (weil eher additive synth)
https://www.youtube.com/watch?v=oKv9S6mxnXE


Notation durchgehen.

Aliasing besprochen?
\href{https://www.youtube.com/watch?v=GBtHeR-hY9Y}{Water experiment}
(youtube \glqq{}The Secret to Levitation\grqq{})

AM, tremolo

Envelopes in pd

FM, vibrato

sounddesign chall.

Hü


\section{AM} % (fold)
\label{sub:AM}

Amplitude Modulation. \glqq{}tremolo\grqq{}


\begin{figure}[H]
	\begin{center}
		\includegraphics[width = 4cm]{img/ringNaive.png}
		\caption{naive Ring modulation}
		\label{fig:name}
	\end{center}
\end{figure}


\begin{figure}[H]
	\begin{center}
		\includegraphics[width = 14cm]{img/simpleEnv.png}
		\caption{caption}
		\label{fig:name}
	\end{center}
\end{figure}

\textbf{Multiplying two signals in the time domain is equvalent to convolution in the frequency domain and vice versa.
}

Ringmodulation = Bipolar,
AM = Unipolar

\section{FM} % (fold)
\label{sub:FM}

Frequency Modulation. \glqq{}Vibrato\grqq{}


\begin{figure}[H]
	\begin{center}
		\includegraphics[width = 7cm]{img/FMgeneral.png}
		\caption{The General Idea of FM}
		\label{fig:fmIdea}
	\end{center}
\end{figure}

Naive parameters are ${f_c}$ (Carrier Frequency), ${f_m}$ (Modulator Frequency), and ${A_m}$ (modulation Amount).

\begin{figure}[H]
	\begin{center}
		\includegraphics[width = 10cm]{img/FMnaive.png}
		\caption{Naive Implementation with Direct Parametrization.}
		\label{fig:fmNaive}
	\end{center}
\end{figure}

The output frequencies will be
\begin{equation}
	f_c \pm n \cdot f_m
\end{equation}

Typically, FM is controlled via \textit{Index}, \textit{Ratio}, and fundamental Frequency. The Index, ${I}$ is given by Modulation Depth and Modulator Frequency.

\begin{equation}
I = \frac{A_m}{f_m}
\end{equation}

A more controllable Implementation will generate the naive parameters from a Ratio, ${R}$, the Carrier Frequency and the Index:
\begin{equation}
	f_m = \frac{f_c}{R}
\end{equation}

\begin{equation}
	A_m = \frac{I}{f_m}
\end{equation}

\begin{figure}[H]
	\begin{center}
		\includegraphics[width = 14cm]{img/FMcorrect.png}
		\caption{FM with Index and Ratio}
		\label{fig:fmComplete}
	\end{center}
\end{figure}

\section{Hausübung}
\label{sub:Hausuebung}
Andy Farnell, \href{http://aspress.co.uk/ds/pdf/pd_intro.pdf}{pd intro} chapter 6, lesen
