\addsec{Ehrenwörtliche Erklärung}
\label{erklaerung}
% \begin{flushleft}
% 	test
% \end{flushleft}

\begin{flushleft}
Ich versichere, dass 
\end{flushleft}

\begin{flushleft}
- ich diese Bachelorarbeit selbständig verfasst, andere als die angegebenen Quellen und Hilfsmittel nicht benutzt und mich sonst keiner unerlaubten Hilfe bedient habe.
\end{flushleft}

\begin{flushleft}
- ich dieses Bachelorarbeitsthema bisher weder im Inland noch im Ausland einem Begutachter/ einer Begutachterin zur Beurteilung oder in irgendeiner Form als Prüfungsarbeit vorgelegt habe.	
\end{flushleft}

\begin{flushleft}
- diese Arbeit mit der vom Begutachter/von der Begutachterin beurteilten Arbeit übereinstimmt. \\[1.5cm]	
\end{flushleft}
% - diese Arbeit mit der vom Begutachter/von der Begutachterin beurteilten Arbeit übereinstimmt. \\
% \\[1.5cm]
Datum:	\hrulefill\enspace Unterschrift: \hrulefill
\\[3.5cm]

% \newpage
% \addsec{Danksagungen}
% \label{danksagungen}
% Weit hinten, hinter den Wortbergen, fern der Länder Vokalien und Konsonantien leben die Blindtexte. Abgeschieden wohnen Sie in Buchstabhausen an der Küste des Semantik, eines großen Sprachozeans. Ein kleines Bächlein namens Duden fließt durch ihren Ort und versorgt sie mit den nötigen Regelialien. Es ist ein paradiesmatisches Land, in dem einem gebratene Satzteile in den Mund fliegen. Nicht einmal von der allmächtigen Interpunktion werden die Blindtexte beherrscht – ein geradezu unorthographisches Leben. Eines Tages aber beschloß eine kleine Zeile Blindtext, ihr Name war Lorem Ipsum, hinaus zu gehen in die weite Grammatik. Der große Oxmox riet ihr davon ab, da.

