\chapter{Introduction}
\label{introduction}

\section{Message Domain/Signal Domain}

\section{What is aliasing?}
Aliasing in audio means problems caused by signals that exceed the nyquist rate.\\
The nyquist rate,let's call it $f_n$ for now, is defined by the half of the sample rate ($f_s$). So, 
\begin{equation}
	f_n=\frac{f_s}{2}
\end{equation}
A digital system can only describe signals up to his nyquist rate. If we try to make signals higher than this frequency, we will fail and encounter strange effects.\\
Visually speaking, frequencies higher than nyquist fold back. So, let's assume we have a sampling rate of 100Hz. Nyquist would be at 50Hz. If we try to synthesize a sine wave with 51Hz, what we will get is a 49Hz one. If we try to make a 52Hz one, we will get 48Hz. So you see, it simply folds back.



\section{Scaling and Mapping Signals}
It is an important skill to be able to scale signals from one range to another. We need it a lot and we will be able to think about signals more easily f we mastered this task. It's actually quite simple, we just have to imagine the signals visually.\\
So what exactly do we have to do here? If we 


\section{What's DC-Offset?}

\section{What's an Impulse?}

\section{How to describe audio mathematically}


% \addsec{Was passiert in Epro und wofuer brauche ich es?}

