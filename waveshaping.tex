\chapter{Sampling, Waveshaping, und nicht Linearität}
\label{Waveshaping}

\begin{center}
\begin{figure}[h!]
\tikzset{concept/.append style={fill={none}}}
\begin{tikzpicture}
  \path[mindmap,concept color=black,text=black]
    node[concept] {Sampling}
    [clockwise from=0]
    child[concept color=red!50!black] {node[concept] (rec) {recording} }
    child[concept color=green!80!black] {node[concept] (pla) {playback} }
    %   [clockwise from=90]
    %   child { node[concept] {algorithms} }
    %   child { node[concept] {data structures} }
    %   child { node[concept] {pro\-gramming languages} }
    %   child { node[concept] {software engineer\-ing} }
    % }
    child[concept color=blue] { node[concept] (ram) {in RAM vs. from Disk}[clockwise from=-30]}
    child[concept color=red] { node[concept] (gra) {granular} }
    child[concept color=orange] { node[concept] {Waveshaping and Distortion} };



\begin{pgfonlayer}{background}
	% \draw [draw=green,fill=black, decorate,decoration=circle connection bar]
    \draw [circle connection bar]
    % \path (rec) to[circle connection bar] (pla)
      (rec) edge (pla)
      (gra) edge (pla)
      (ram) edge (pla)
      ;
  \end{pgfonlayer}


\end{tikzpicture}
\caption{Lecture Contents}
\end{figure}
\end{center}



\section{Notizen}

Waveshaping generell.

Evt. Subtraktiver Synth am ende.

Zusammenhang waveshaping, distortion. lookup table, sampling, wavescanning.

Waveshaping = modulation \cite[p.~257]{farnell_designing_2010}

chebyshev

\glqq{}intermod. distortion \grqq{} , vgl. miller puckette, \\
http://msp.ucsd.edu/techniques/latest/book-html/node78.html

middle term! ( (a+b)**2 = a**2 +2ab +b**2 )




beispiele:
sinus
vllt:\\
dyn. non-linear functions




\section{Uebersicht}
Zunächst: Sampling, dann lineare transfer function.

\begin{enumerate}
	\item übersicht: fragen wies geht, wie ist es mit Max/MSP?
	\item HÜ ankündigen
	\item neue abstraction: z-1
	\item Raetsel
	\item Wo letztes mal stehngeblieben? Poly syth fertig?
	\item linearität erklären. Nun Nichtlinearität.
	\item wavetable/lookup wiederholung
	\item Lookup als waveshaping/distortion
	\item waveshaping durch processing vs. durch lookuptable
	\item durchrechnen
	\item chebyshev
	\item Sampling (Ram nicht Ram)
	\item send and receive von GUI f. Hausübung
\end{enumerate}

\section{Waveshaping}
Wikipedia quote, page ``wavesahper'':\\
\glqq{}The mathematics of non-linear operations on audio signals is difficult, and not well understood.\grqq{}
Waveshaping means distortion. It add overtones.


\subsection{The simples case: a linear Transfer function. } % (fold)
\label{sub:linearTrans}
See \ref{fig:linfunct}. A linear transfer function is used as a lookup table for a sinosoidal input.

\begin{figure}[H]
	\begin{center}
		\includegraphics[scale = 1]{img/waveShapingVisual.png}
		\caption{Linear Transfer function}
		\label{fig:linfunct}
	\end{center}
\end{figure}
A transfer function in the sense of a waveshaper (a ``transfer function'' might also mean frequency response in other contexts) is a simple look-up function. \
Waveshaping means to use an input wave \textit{look up} values in a table or function. A linear transfer function, let's call it $l$, results in no change, since, by definition, it returns $l(x)=x$. \
This means, that whatever value we pass in, we get the same value out.\
Non-liner transfer functions behave differently. They map they input to other values, such as $f(x) = x^2$. It may seem trivial, but if we put 2 into $f$ we get 4 as an output. If we now plot this function we get:

\begin{figure}[H]
	\begin{center}
		\includegraphics[width = 14cm]{img/squareFunction.png}
		\caption{The function $f(x)=x^2$}
		\label{fig:square}
	\end{center}
\end{figure}

% subsection subsection_name (end)




\subsection{Simple non-linearity: \(X ^2\)} % (fold)
\label{sub:nonLinearTrans}

A 2nd order polynomial shall be analyzed. The function

\begin{equation}
f(x) = x ^ 2
\end{equation}
 is used and also plotted in figure \ref{fig:square}.\\


% \fbox{
%   \parbox{\textwidth}{
%     Ein polynom geradzahliger ordnung \(n\) als Transferfunktion produziert immer alle geradzahligen obertöne von \(n\) bis 0, exklusive der grundfrequenz (da ja auch ungerade, 1 ist eine ungerade zahl).\\
% Ein polynom ungeradzahliger ordnung \(n\) als Transferfunktion produziert immer alle ungeradzahligen obertöne von \(n\) bis 1, also der grundfrequenz.
%   }
% }
We can simply plot what happened is we apply the function before trying to understand analytically:


\begin{figure}[H]
	\begin{center}
		\includegraphics[width = 14cm]{img/sinSquared.png}
		\caption{Applying the square function to an input sine wave.}
		\label{fig:sinSquared}
	\end{center}
\end{figure}


Weird, the input seems to double in frequency. Let's try to understand what's happening.\

So we calculate what happens if we send a cosine through this function, so let's take:

\begin{equation}
x = cos(\omega)
\end{equation}
wit arbitrary $\omega$. We can just ignore $\omega$ here for a while. Usually, there should be some indexing variable in the cosine function if we want to describe an oscillator that moves over time, but let's also skip that.\\
So applying our square function we of course get:
\begin{equation}
y = cos(\omega)^2
\end{equation}
This again results in:
\begin{equation}
y = cos(\omega) \cdot cos(\omega)
\end{equation}
So far so trivial. Note that a multiplication of two oscillators is called \textit{Amplitude Modulation} (actually, in this case we encounter ``Ring Modulation'', but let's ignore that also), and we know things about Amplitude modulation, namely:\

\fbox{
  \parbox{\textwidth}{
  When multiplying two oscillators, we get sum and difference of the two input frequencies. (And the whole output is attenuated by 6dB)
  }}
The above statement in equation form:
\begin{equation}
	cos(a)\cdot cos(b) = \frac{cos(a+b) + cos(a-b)}{2}
\end{equation}

We could also have looked up this \textit{trigonometric identity}.
This means for our experiment with our cosine squared:

\begin{equation}
y = \frac{cos(\omega+\omega) + cos(\omega-\omega)}{2}
\end{equation}
So:
\begin{equation}
y = \frac{cos(2 \cdot \omega) + cos(0)}{2} = \frac{cos(2 \cdot \omega )}{2}+\frac{1}{2}
\end{equation}

We arrive at the same result!
\textbf{But is this true for every input? That would mean we just built a frequency shofter, did we? No.} Waveshaping is much more complicated, which is immediately obvious when we try to do the same t two oscillators:
\begin{equation}
x = cos(\omega_1)+cos(\omega_2)
\end{equation}
then
\begin{equation}
y = (cos(\omega_1)+cos(\omega_2) ) ^2
\end{equation}
\begin{equation}
y = cos(\omega_1)^2+cos(\omega_2)^2+2\cdot cos(\omega_1) \cdot cos(\omega_2)
\end{equation}

And finally:
\begin{equation}
	y = \frac{cos(2 \cdot \omega_1)}{2} + \frac{1}{2} + \frac{cos(2 \cdot \omega_2)}{2} +\frac{1}{2} + 2 \cdot (\frac{cos(\omega_1+\omega_2) + cos(\omega_1-\omega_2)}{2})
\end{equation}



\subsection{Why is Waveshaping useful?}
The output spectrum is dependent on the input amplitude. This makes it easy to create complex involving spectra.

Wieso ist waveshaping praktisch? Amplitudenabhängigkeit d. spectrums.
Wieso ist waveshaping verwandt mit modulation, sowohl FM, PM als auch AM? Am: siehe \(x^2\). FM: siehe \(f(x)=cos(x)\). Auch kann letztendlich eine transferkuntion in cos/sin bestandteile zerlegt werden um zu einer menge an frequenzmodulationen anzukommen, bzw kann der wavetable als polynom angenähert(o. Taylor entwicklung) werden um bei AM anzukommen.



\section{Sampler}

Sampling von d. Festplatte vs. vom RAM.\\
Zusammen bauen. Geschwindigkeit modulieren etc.\\
Evtl. \texttt{moresampling.pd}, granular sampling kurz erklären.


\begin{figure}[H]
	\begin{center}
		\includegraphics[scale = 1]{img/sampler.png}
		\caption{simpleSampler}
		\label{fig:simpleSampler}
	\end{center}
\end{figure}

\begin{figure}[H]
	\begin{center}
		\includegraphics[scale = 1]{img/soundFileToRam.png}
		\caption{sound in Ram}
		\label{fig:soundRam}
	\end{center}
\end{figure}


\begin{figure}[H]
	\begin{center}
		\includegraphics[scale = 1]{img/RamFilePlayback.png}
		\caption{RamFilePlayback}
		\label{fig:RamFilePlayback}
	\end{center}
\end{figure}



\begin{figure}[H]
	\begin{center}
		\includegraphics[width = 14cm]{writingAudio.png}
		\caption{writing Audio to disk}
		\label{fig:writing}
	\end{center}
\end{figure}


\begin{figure}[H]
	\begin{center}
		\includegraphics[scale = 0.6]{img/grain.png}
		\caption{moreSamplimg.pd, a simplified version of granular sampling}
		\label{fig:name}
	\end{center}
\end{figure}



\section{Hausübung}
\subsection{Testmodul}

baue ein audio Testmodul mit folgener spezifikation:\\
\begin{itemize}
	\item Ein audio output
	\item verschiedene klangquellen wählbar:
		\begin{enumerate}
			\item White Noise
			\item Sinus (freq. einstellbar)
			\item soundfile (file wählbar)
		\end{enumerate}
	\item GUI
	\item verfügbar(in eurem pfad, und jederzeit abrufbar als abstraction)
	\item output pegel sichtbar (level meter)
\end{itemize}


\begin{figure}[H]
	\begin{center}
		\includegraphics[scale = 1]{img/audioTester.png}
		\caption{audioTester.pd, zu bauen als Hausübung}
		\label{fig:audiotester}
	\end{center}
\end{figure}

\subsection{Distortion}
Baue eine besonders wohlklingende Distortion, inkl. User interface.

