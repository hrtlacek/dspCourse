\chapter{Results}
\label{results}
The final application looks like depicted in fig. \ref{fig:app} and is accompanied by the mentioned synthesis modules.

\begin{figure}[h]
	\begin{center}
		\includegraphics[width = 14cm]{img/AppFin.png}
		\caption{The final GUI of the analysis application}
		\label{fig:app}
	\end{center}
\end{figure}

In general, the analysis/synthesis process works very well for many sounds. The proposed system has the potential to be one good tool in a sound designers tool set, but will not solve the problem of realistically modeling just \textit{any} given body if just used on its own. As \citep{farnell_designing_2010} shows in many of the synthesis algorithms he describes, a small number of resonance modes in combination with other techniques suffices for most problems. But finding these few modes is dramatically eased by use of the proposed tool inside Max or similar tools. 
This document is accompanied by two synthesized files:
\begin{description}
	\item [Synthesized trash can:] trashBucketSynthesized.wav
	\item [Synthesized wooden table:] tableSynthesized.wav
\end{description}

Both source impulse responses have been generated using a drumstick.
The original impulse responses are also included:
\begin{itemize}
	\item trashBucketIR.wav  
	\item TableIR.wav
\end{itemize}
Therefore the reader may judge for himself, how close the application can get to a real sound. Both synthesized sounds have only used the synthesis capabilities of the analysis patch and no further processing has been applied.\\
In general, the modular nature of the application allows a skilled user to exchange any part of the application. But even a non-programming sound designer can take great advantage of the modular structure and the verbosity of the GUI. A sound designer might just look at the spectrum to draw his own conclusions, or at the autocorrelation etc. Or he or she might use the list of detected frequencies to go their own way from there. Any point in the processing can be used as a starting point for own synthesis/analysis methods, if the provided ones do not suffice or are unintuitive for the particular user.\\
A by-product of the application is the extraction of the exciter signal estimation. It is not used by the algorithm although it is recorded and can be inspected by the inclined user. In general the excitation signal has not received much attention in this work, since it's composition is assumed to be a lot simpler than the resonator. Also, it is rather likely that a potential user is eager to play the trash bucket arco, or experiment with different excitation signals for a given resonator. A rather untypical use would be to extract a excitation signal from one impulse response (which might not be even near an impulse response and might contain an interesting exciter) to send it into a different resonator. This setup can be achieved also with this application if the excitation signal is saved. 







