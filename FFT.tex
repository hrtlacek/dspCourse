%!TEX root = main.tex

\chapter{Fourier Transform}
\label{Fourier Transformation}

\begin{figure}[H]
	\begin{center}
		\includegraphics[width = 14cm]{img/aphexFFT.png}
		\caption{Image in the spectrogram in a track by ``Aphex Twin''.}
		\label{fig:Aphex}
	\end{center}
\end{figure}




\begin{center}
\begin{figure}[h!]
\tikzset{concept/.append style={fill={none}}}
\begin{tikzpicture}
  \path[mindmap,concept color=black,text=black]
	node[concept] {Fourier Tansform}
	[clockwise from=0]
	% child[concept color=green!50!black] {
	% node[concept] {schuluebung v. letzem Mal}
	% }
	% child[concept color=green!50!black] {node[concept] {pd parxis}
	% 	child[concept color=green!50!black] {node[concept] {send/receive}}
	% 	child[concept color=green!50!black] {node[concept] {initialisierung}}
	% 	}
	% child[concept color=blue] {
	% node[concept] {hauptteil FFT}
	% [clockwise from=-90]
		child[concept color=orange] { node[concept] {theory} }
		child[concept color=red] { node[concept] {practice}
			child[concept color=red] { node[concept] {spectrum display} }
			child[concept color=red] { node[concept] {fft filter}
				child[concept color=red] { node[concept] {random filter} }
				}
			child[concept color=red] { node[concept] {fft reverb} }
			child[concept color=red] { node[concept] {pitch shift} }
		}


;
\end{tikzpicture}
\caption{Lecture Contents: Fourier Transform}
\end{figure}
\end{center}

\section{Further Information}

\begin{itemize}
	\item \link{https://www.youtube.com/watch?v=-IJuqR6nz\_Q}{complex Numbers}
	\item \link{https://www.youtube.com/watch?v=EIstpPXKWng}{The Imaginary Number}
\end{itemize}

vorbereitung:\\
Eulers identity, complex numbers.



Initialisierung
Fourier transformation.
Spectral filter
Spectral Reverb, delay.
Windowing
Convolution,
evtl. cross correlation
freq. crossover
spectral synyth,
spectraum display.


% evtl auch:
% \begin{itemize}
% 	\item send / receive, send / receive bei gui objekten.
% 	\item initialisierung
% 	\item
% \end{itemize}


Diskretes Signal -> Periodisches Spectrum \\
Periodisches Signal -> Diskretes Spectrum

gute referenz:
http://jackschaedler.github.io/circles-sines-signals


% fftuebung
% beiwerk:
	% initialisierung
	% send/receive
% schuluebung von letzem mal
% hauttteil fft
	% theorie	> geschichte
	% praxis
		% spectrum display
		% fft filter
			% nomral
			% random
		% fftdelay
		% fft reverb
		% pitch shift


\section{Fourier Transformation}

Geschichte:
Bernoulli, Euler, Gauß, Fourier

Eulers identität:
\begin{equation}
	e ^{ix} = cos(x)+i \cdot sin(x)
\end{equation}

Fourier Transformation:\\
\begin{equation}
	X(f)= \mathcal{F} \{x(t)\} = \int_{-\infty}^\infty \! x(t) e^{-j2\pi ft} \, \mathrm{d}t
\end{equation}

Inverse Fourier Transformation:\\
\begin{equation}
	x(t)= \mathcal{F}^{-1} \{X(f)\} = \int_{-\infty}^\infty \! X(f) e^{j2\pi ft} \, \mathrm{d}f
\end{equation}

The discrete Fourier Transform (DFT) is defined as:
\begin{equation}
	x_f(m) = \sum_{n=0}^{N-1} x(n)\cdot e^{-i 2 \pi \frac{n m}{N} }
\end{equation}
\newpage
As python code this could look like this:
\lstinputlisting[language=Python,caption={DFT in python},captionpos=b]{code/fourierTransform.py}
Beware that the python code above implements the formula directly. It is \textit{very} slow. Typically, a different algorithm is used, such as an implementation of the Fast Fourier Transform (FFT). In python, there are several ways to compute an FFT in an efficient way such as \texttt{numpy.fft()}.

\begin{figure}[h]
	\begin{center}
		\includegraphics[width = 14cm]{img/spectralFilter.png}
		\caption{spectralFilter.pd}
		\label{fig:spectralFilter}
	\end{center}
\end{figure}

\begin{figure}[h]
	\begin{center}
		\includegraphics[width = 14cm]{img/showspectrum.png}
		\caption{showspectrum}
		\label{fig:showspectrum}
	\end{center}
\end{figure}
