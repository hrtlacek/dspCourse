%Dokumentklasse
\documentclass[a4paper,11pt]{scrreprt}
\usepackage[left= 3.5cm,right = 3cm, bottom = 3.5 cm, top = 3 cm]{geometry}
\usepackage[onehalfspacing]{setspace}
% ============= Packages =============

% Dokumentinformationen
\usepackage[
	pdftitle={EPRO_UE_SCRIPT_001},
	pdfsubject={},
	pdfauthor={Patrik Lechner},
	pdfkeywords={}
	pdftex=true, 
	colorlinks=true,
 	breaklinks=true,
	citecolor=black,
	linkcolor=black,	
	menucolor=black,	
	urlcolor=black
]{hyperref}



% Standard Packages
\usepackage[utf8]{inputenc}

% deutsch
% \usepackage[ngerman]{babel}
% \usepackage[ngerman, num]{isodate}

% english
\usepackage[english]{babel}

\usepackage[T1]{fontenc}
\usepackage{graphicx}
\usepackage{graphicx, subfigure}
%setze pfad fuer Bilder
\graphicspath{{img/}}
\usepackage{fancyhdr}
\usepackage{lmodern}
\usepackage{color}
\usepackage{transparent}


% Citation style
\usepackage[comma,authoryear]{natbib}
%\usepackage{natbib}

% zusätzliche Schriftzeichen der American Mathematical Society
\usepackage{amsfonts}
\usepackage{mathtools}

% BlockDiagram Drawing Package
% ---tikz
% \usepackage{tikz}
% \usepackage{pgfplots}
% ---diagraph?


% -- Settings für Code abbildungen
% \usepackage{listings}

% \definecolor{dkgreen}{rgb}{0,0.6,0}
% \definecolor{gray}{rgb}{0.5,0.5,0.5}
% \definecolor{mauve}{rgb}{0.58,0,0.82}

% \lstset{frame=tb,
%   language=Java,
%   aboveskip=3mm,
%   belowskip=3mm,
%   showstringspaces=false,
%   columns=flexible,
%   basicstyle={\small\ttfamily},
%   numbers=none,
%   numberstyle=\tiny\color{gray},
%   keywordstyle=\color{blue},
%   commentstyle=\color{dkgreen},
%   stringstyle=\color{mauve},
%   breaklines=true,
%   breakatwhitespace=true
%   tabsize=3
% }




% Setze arial font
\renewcommand*{\familydefault}{\sfdefault}

% FH-grünBlau
\definecolor{FH}{rgb}{0.10, 0.57, 0.68}


% nicht einrücken nach Absatz
\setlength{\parindent}{0pt}


% ============= Kopf- und Fußzeile =============
\pagestyle{fancy}
%
\lhead{}
\chead{}
\rhead{\slshape \leftmark}
%%
\lfoot{}
\cfoot{}
\rfoot{\thepage}
%%
\renewcommand{\headrulewidth}{0.4pt}
\renewcommand{\footrulewidth}{0pt}

% ============= Package Einstellungen & Sonstiges ============= 

%Besondere Trennungen
\hyphenation{De-zi-mal-tren-nung St-rei-fen-licht-scan-nern}

%römische Aufzählungen mit \RM{Zahl}
\newcommand{\RM}[1]{\MakeUppercase{\romannumeral #1}}


% ============= Dokumentbeginn =============

\begin{document}

\pagestyle{empty}
%!TEX root = main.tex

\pagestyle{empty}
\begin{center}
\begin{tabular}{p{\textwidth}}


\begin{center}
\includegraphics[scale=0.7]{img/FHlogo.jpg}
\end{center}

\\

\begin{center}
\Huge{{\color{FH}{\fontsize{24}{48} \selectfont Signal Processing\\}}}
\end{center}

\\

\\

\begin{center}
Bachelor Studiengang Medientechnik\\
Fachhochschule St.Pölten
\end{center}

\\
\begin{center}
Patrik Lechner
\end{center}

\\

\begin{center}
\large{Wien, \today}
\end{center}


\end{tabular}
\end{center}

% \part im Inhaltsverzeichnis nicht nummerieren
\makeatletter
\let\partbackup\l@part
\renewcommand*\l@part[2]{\partbackup{#1}{}}

%Seitennummerierung neu beginnen, Zahlen [arabic], röm.Zahlen [roman,Roman], Buchstaben [alph,Alph]
\pagenumbering{Roman}

% \newpage

% \addsec{Ehrenwörtliche Erklärung}
\label{erklaerung}
% \begin{flushleft}
% 	test
% \end{flushleft}

\begin{flushleft}
Ich versichere, dass 
\end{flushleft}

\begin{flushleft}
- ich diese Bachelorarbeit selbständig verfasst, andere als die angegebenen Quellen und Hilfsmittel nicht benutzt und mich sonst keiner unerlaubten Hilfe bedient habe.
\end{flushleft}

\begin{flushleft}
- ich dieses Bachelorarbeitsthema bisher weder im Inland noch im Ausland einem Begutachter/ einer Begutachterin zur Beurteilung oder in irgendeiner Form als Prüfungsarbeit vorgelegt habe.	
\end{flushleft}

\begin{flushleft}
- diese Arbeit mit der vom Begutachter/von der Begutachterin beurteilten Arbeit übereinstimmt. \\[1.5cm]	
\end{flushleft}
% - diese Arbeit mit der vom Begutachter/von der Begutachterin beurteilten Arbeit übereinstimmt. \\
% \\[1.5cm]
Datum:	\hrulefill\enspace Unterschrift: \hrulefill
\\[3.5cm]
% \newpage
% \addsec{Abstract}
\label{sec:zusammenfassung}
This document tries to show a possible way of estimating bandpass filter coefficients from a given impulse response in a semi-automatic fashion. This impulse response is assumed to be generated by striking a rigid body, and the resulting set of estimated coefficients are supposed to help a sound designer to develop sounds based on a physical body. Max/MSP is used for real time synthesis. Also Max/MSP in combination with Python/numpy is used for the analysis.

% \minisec{Abstract}
% \label{abstract}


\newpage
\pagestyle{fancy}
%Inhaltsverzeichnis
\tableofcontents

\newpage
%Seitennummerierung neu beginnen, Zahlen [arabic], röm.Zahlen [roman,Roman], Buchstaben [alph,Alph]
\pagenumbering{arabic}

% pagestyle für gesamtes Dokument aktivieren
\pagestyle{fancy}

\newpage
\include{05_Filters}

% \include{05_method}

% \chapter{Stand der Technik}
\label{cha:stand_der_technik}

\section{Klassifizierung von Beispielen}
\label{sec:klassifizierung}

\section{Themenbezogene Veröffentlichungen}
\label{sec:themenbezogene_veroeffentlichungen}

% \chapter{Entwicklung der Sensorik}
\label{chap:entwicklung}


\section{Grobkonzept der Sensorik}
\label{sec:grobkonzept}

% \chapter{Ergebnisse}
\label{sec:ergebnisse}


% \chapter{Discussion}
\label{sec:discussion}
Several potential future improvements became obvious during the development of this document and the programs. Mainly due to reflections, many bodies show a comb filter like spectrum. Synthesizing these spectra with  many resonance filters is a waste of CPU cycles. A possible future approach could be to first do a cepstral analysis on the impulse response. This analysis can provide a parametrization for a comb filter. Moreover, the inverse of the comb filter could be applied to the impulse response to then analyze the spectrum for the resonance filters. In the re-synthesis stage, the results of the resonance filter are fed through the comb filter(s) to end up with a possibly more efficient and controllable model.\\
Hardware GPU acceleration, so shaders, could be used to speed up the analysis process, since GPUs are optimized for matrix calculations and the peak finding problem is actually a 2 dimensional one although it hasn't been treated as such in this document. If the spectrum over time is written to a texture and properly preprocessed, even regular blob-detection programs could be used.\\
Another possible future improvement is to rather let the analysis communicate with the resonator(s) via UDP/OSC, so other applications would have direct access to the data without the need of saving a text file. \\
Lastly, the decision, what frequencies might be the most important ones, was a central point of this document. And the algorithm was designed having a very simple, specific answer to the problem in mind: the loudest frequencies are the most important ones. This, however, is wrong or at least inaccurate, since the question really is a psychoacoustic one. For example masking of certain frequencies by others comes into play at least if one tries to track very many frequencies. In that case, an algorithm trying to take psychoacoustics into account might significantly lower the number of needed band pass filters, choosing the most important ones based on gain and proximity to other frequencies and their gains. Also a K-weighting curve could be used to pre-process the spectrum in order to amplify frequencies a human is more sensitive to. This would definitely be an easy step towards picking important frequencies instead of loud ones.



%Verzeichnis aller Bilder
\newpage
\listoffigures

%Literaturverzeichnis
\newpage
% unsrtdin
\bibliographystyle{apalike}
%\bibliography{BAKK2.bib}
\bibliography{BAKKlibrary2}
\end{document}
